\documentclass[12pt]{article}
\usepackage[english]{babel}
\usepackage[utf8]{inputenc}
\usepackage[T1]{fontenc}
\usepackage{amsmath}
\usepackage{graphicx}
\usepackage[colorinlistoftodos]{todonotes}
\usepackage{listings}
\usepackage{enumitem}
\usepackage{listingsutf8}
\usepackage{xparse}
\usepackage[hmargin=2cm]{geometry}
\usepackage{color} 
\NewDocumentCommand{\codeword}{v}{%
\texttt{\textcolor{blue}{#1}}%
}
\definecolor{codegreen}{rgb}{0,0.6,0}
\definecolor{codegray}{rgb}{0.5,0.5,0.5}
\definecolor{codepurple}{rgb}{0.58,0,0.82}
\definecolor{backcolour}{rgb}{0.95,0.95,0.92} 
\lstdefinestyle{mystyle}{
    backgroundcolor=\color{backcolour},   
    commentstyle=\color{codegreen},
    keywordstyle=\color{magenta},
    numberstyle=\tiny\color{codegray},
    stringstyle=\color{codepurple},
    basicstyle=\footnotesize,
    breakatwhitespace=false,         
    breaklines=true,                 
    captionpos=b,                    
    keepspaces=true,                 
    numbers=left,                    
    numbersep=5pt,                  
    showspaces=false,                
    showstringspaces=false,
    showtabs=false,                  
    tabsize=2
}
\lstset{style=mystyle}
\lstset{inputencoding=utf8/latin1}
%Para mostrar el código bonito 
\lstset{language=C++} 
\lstdefinestyle{customc}{
  belowcaptionskip=1\baselineskip,
  breaklines=true,
  frame=L,
  xleftmargin=\parindent,
  language=C++,
  showstringspaces=false,
  basicstyle=\footnotesize\ttfamily,
  keywordstyle=\bfseries\color{green!40!black},
  commentstyle=\itshape\color{purple!40!black},
  identifierstyle=\color{blue},
  stringstyle=\color{orange},
}

\begin{document}
\begin{titlepage}
\newcommand{\HRule}{\rule{\linewidth}{0.5mm}}
\center
\textsc{\LARGE Universidad de Granada}\\[1.5cm] % Name of your university/college
\textsc{\Large Modelos de computación}\\[0.5cm] % Major heading such as course name
\HRule \\[0.4cm]
{ \huge \bfseries Práctica de Lex}\\[0.4cm] % Title of your document
\HRule \\[1.5cm]
\begin{minipage}{0.4\textwidth}
\begin{flushleft} \large
\emph{Autora:}\\
Elena Merelo Molina \textsc{} % Your name
\end{flushleft}
\end{minipage}
~
\begin{minipage}{0.4\textwidth}
\begin{flushright} \large
\emph{} \\
\textsc{} % Supervisor's Name
\end{flushright}
\end{minipage}\\[2cm]
{\large Diciembre de 2018}\\[2cm] % Date, change the \today to a set date if you want to be precise
\includegraphics[scale=0.5]{./logo.png}
\vfill % Fill the rest of the page with whitespace
\end{titlepage}


\section{Explicación de mi solución}
Dado un fichero con ránkings de atletismo, se busca encontrar los tres mejores atletas en sus categorías(considerando las pruebas de velocidad, fondo y medio fondo, que son del tipo número + metros (100 metros o kilómetros). El fichero tiene el siguiente formato: primero pone el nombre de la prueba: cuatrocientos metros lisos, mil quinientos,... Y luego ya la clasificación en sí, empezando por el que tiene el mínimo tiempo, siendo ésta la primera columna, luego en la segunda (solo para el caso de 100 y 200 metros) la velocidad del viento, en la tercera (para el resto de pruebas que no son 100m o 200m es la segunda) el nombre del atleta, en la cuarta su primer apellido, posteriormente su código de identificación en la RFEA, calle por la que corría y por último en qué año tuvo lugar la carrera. Mi programa pues analiza un archivo que se le pase y de todos los atletas de cada categoría obtiene los tres mejores. En el fichero mejores.l se ve claro qué hace cada cosa.
Para ver que funciona, simplemente habría que hacer bash exe.sh (habiéndole dado previamente permisos de ejecución).

\end{document}